
\documentclass[11pt,a4paper]{article}
\usepackage[T1]{fontenc}
\usepackage[utf8]{inputenc}
\usepackage{authblk}
\usepackage[english]{babel}
\usepackage{fancyhdr}
\usepackage{subfig}
\usepackage{floatrow}
\usepackage{float}
\usepackage{todonotes}
\usepackage{amsmath}
\usepackage{amssymb}
\usepackage{slashed}
\usepackage{graphicx}
\usepackage{todonotes}
\usepackage[toc,page]{appendix}
\usepackage{hyperref}
\usepackage{placeins}
\usepackage{cleveref}
\hypersetup{
	colorlinks,
	citecolor=purple,
	filecolor=black,
	linkcolor=blue,
	urlcolor=red
}




\newcommand*\samethanks[1][\value{footnote}]{\footnotemark[#1]}
\title{Intention to submit a proposal to ERC}
\date{}
\author[1]{M. C. Kunkel\thanks{Contact person, email: m.kunkel@fz-juelich.de}}
 
\renewcommand\Authands{, }

\newlength{\figwidth}
\setlength{\figwidth}{0.9\columnwidth}

\newlength{\qfigheight}
\setlength{\qfigheight}{0.25\textheight}

\newlength{\hfigheight}
\setlength{\hfigheight}{0.5\textheight}

\def\piz{\pi^{0}}
\def\pizT{$\pi^{0} \ $}
 \def\pizDal{$\pi^{0} \rightarrow e^+e^- \gamma  $}
 
\def\etaT{$\eta $}
 \def\etaDal{$\eta \rightarrow e^+e^- \gamma  $}
 
\def\omT{$\omega  $}
 \def\omDal{$\omega \rightarrow e^+e^- \piz $}
 
\def\etaP{\eta^{\prime}}
\def\etaTP{$\eta^{\prime}  $}
 \def\etaPDal{$\eta^{\prime} \rightarrow e^+e^- \gamma  $}

 \def\phiT{$\phi  $}
 \def\phiDal{$\phi \rightarrow e^+e^- \eta  $}
 \def\phiDalT{\phi \rightarrow e^+e^- \eta  }
 
 \def\epemT{$ e^+e^-  $}
  \def\pipiT{$\pi^+\pi^-$}
 \def\epem{e^+e^-}

 \def\phiPR{$ep\to e'p \phi \rightarrow p e^+e^- \eta$}
 \def\etaPR{$ep\to e'p \etaP \rightarrow p e^+e^- \gamma$}
 
\def\grpath{figures}
\newcommand{\abbr}[1]{\textsc{\texttt{#1}}}
%\input{variables}
% Document starts
\begin{document}
\maketitle

\begin{abstract}
I want to write a proposal for a grant from the European Research Council (ERC). I am seeking approval from the Vorstand to have this grant in conjunction with the research activities of IKP-1 at Forschungszentrum J\"ulich. The proposal will focus on maintaining a program related to measuring transition form factors (TFF's), which are an important tool to understand the internal structure of matter and testing the limits of the Standard Model of particle physics. This topic has been in the portfolio of IKP-1 for many years and measurements have been carried out successfully at COSY. 

There is a need for new experimental data involving heavier mass states of particles, in addition to higher precision of the measurement. The data to be analyzed will be collected at the newly upgraded CLAS detector located at the Thomas Jefferson National Facility in Newport News Virgina, U.S.A. The CLAS detector will capable of measuring TFF's with high statistics and excellent precision.

IKP-1 is the only German institute that is a formal CLAS institute. Experience with CLAS has already been established at IKP-1 with data from the former CLAS detector setup. The proposal aims to extend the current CLAS actives at IKP-1 with focus on the transition form studies mentioned above. The new CLAS detector is scheduled to take data in the year 2017 and a proposal for these types of measurements is already under review in CLAS. The estimated time-frame from data taking to publication is in the order of 3 years.

\end{abstract}

\end{document}
